% Auf ungerader Seite starten
\cleardoublepage
%%%%%%%%%%%%%%%%%%%%%%%%%%%%%%%%%%%%%%%%%%%%%%%%%%%%%%%%%%%%%%%%%%%%%%%%%%%%%%%%%%%%%%%%%
% Einleitung
%%%%%%%%%%%%%%%%%%%%%%%%%%%%%%%%%%%%%%%%%%%%%%%%%%%%%%%%%%%%%%%%%%%%%%%%%%%%%%%%%%%%%%%%%

\chapter{Abschlussarbeiten am ITS}
\label{Chapter:Aufbau der Arbeit}

Diese Vorlage dient zur Erstellung wissenschaftlicher Abschlussarbeiten am Institut für Thermische Strömungsmaschinen (ITS) und stellt eine Zusammenfassung der Meinungen aller wissenschaftlicher Mitarbeiter dar. Zu manchen Punkten kann der Betreuer eigene Meinungen haben. Es existiert eine äquivalente Word-Vorlage, deren Benutzung mit dem Betreuer abzusprechen ist. 

In diesem Kapitel wird auf den allgemeinen Aufbau einer Abschlussarbeit eingegangen. In Kapitel~\ref{CHAPTER:Hinweise} wird im Detail auf einzelne Aspekte der sprachlichen Gestaltung, Aufbau von Gliederung, Diagrammen, Abbildungen sowie Gleichungen eingegangen.

\section{Allgemeines}
Die Arbeit ist in deutscher Sprache anzufertigen und wird einseitig gedruckt. Der Gesamtumfang beträgt in der Regel für Bachelorarbeiten 30~--~60 und für Masterarbeiten 60~--~80 Seiten. Diese Angaben sind lediglich als Richtlinie zu werten und erfolgen immer in Absprache mit dem Betreuer. Vor Beginn des Schreibens ist die Gliederung mit dem Betreuer abzusprechen. Um den Korrekturaufwand möglichst gering zu halten, ist es sinnvoll, eine Vorkorrektur an einem Teil der Arbeit durchzuführen, bevor die gesamte Arbeit zur Korrektur abgegeben wird. Für die Einarbeitung in das Thema sollten ungefähr zwei Wochen eingeplant und spätestens sechs Wochen vor Abgabe mit dem Schreiben begonnen werden.

Grundsätzlicher Aufbau einer Abschlussarbeit:
\begin{itemize}
\itemsep0em 
	\item ITS Deckblatt
	\item Leerseite
	\item Gegebenenfalls ein Sperrvermerk
	\item Aufgabenstellung
	\item Unterschriebene eidesstattliche Erklärung
	\item Danksagung (freiwillig!)
	\item Evtl. Kurzfassung/Abstract
	\item Inhaltsverzeichnis
	\item Abbildungsverzeichnis
	\item Tabellenverzeichnis
	\item Symbolverzeichnis
	\item \textbf{Hauptteil} (siehe folgende Unterkapitel)
	\item Literaturverzeichnis
	\item Anhang
\end{itemize}

Im Folgenden ist beschrieben, aus welchen Bausteinen der Hauptteil einer Abschlussarbeit besteht, welche Aufgabe diese Bausteine haben und wie sie mit Inhalt zu füllen sind. Um den Leser an einem \glqq roten Faden\grqq~durch die Arbeit zu führen, ist es hilfreich kurze Zusammenfassungen am Ende der Kapitel bzw. Überleitungen zum nachfolgenden Kapitel zu liefern.

\section{Einleitung}
\label{SECTION:Einleitung}

Die Einleitung hat die Funktion zum Thema hinzuführen und liefert Antworten, warum das Thema der Arbeit aus technischer, wissenschaftlicher und evtl. wirtschaftlicher Sicht von Bedeutung ist. Mit Ende der Einleitung muss das Thema der Arbeit klar beschrieben sein. Daher ist der Anfang sehr allgemein gehalten, während zum Schluss das Problem eingegrenzt wird. Hier gilt: Möglichst keine externe Quellen!

\section{Stand der Forschung}

Im Stand der Forschung wird aufgezeigt, welche Arbeiten es zu diesem bzw. ähnlichen Themen gibt und welche Lücke diese Arbeit füllen soll. Hierbei sind vor allem sämtliche Vorgängerarbeiten des ITS ausfindig zu machen. Als Quellen sind bevorzugt Zeitschriftenartikel und Konferenzbeiträge statt Dissertationen anzugeben. Im letzten Abschnitt des Kapitels wird eine konkrete Zielsetzung der Arbeit und die Vorgehensweise erarbeitet.


\section{Grundlagen}

Grundlagen, die zum Verständnis der weiteren Kapitel notwendig sind, werden beschrieben. Daher ist dieses Kapitel möglichst knapp zu gestalten, es handelt sich nicht um ein Lehrbuch! Um Platz zu sparen, können Verweise auf weiterführende Lehrbücher verwendet werden. Zum Beispiel: 
\begin{quote}
	\glqq Im Rahmen dieser Arbeit kommt nur das $k\text{-}\epsilon$-Turbulenzmodell zum Einsatz. Deshalb wird an dieser Stelle auf andere Turbulenzmodelle nicht weiter eingegangen. Diese werden ausführlich in Autor XY et al. (2017) erläutert.\grqq
\end{quote}
Aussagen in diesem Kapitel werden mit Quellenangaben versehen, wenn es sich nicht um allgemeines Wissen im Ingenieurwesen handelt. Herleitungen komplexer Formeln sind nur mit Bedacht zu verwenden. Um zu verhindern, dass der Leser gezwungen ist im Text vor- und zurückzuspringen, ist auf einen logischen Aufbau des Kapitels zu achten. Dies kann zum Beispiel nach Teilgebieten, chronologisch oder nach Komplexität gestaltet sein. Verweise auf noch kommende Textstellen sind zu vermeiden. Besser: \glqq In Abschnitt \ref{SECTION:Einleitung} wurde gezeigt, dass ... \grqq.

\section{Material und Methoden}

Hier wird der experimentelle Aufbau bzw. das verwendete numerische Modell beschrieben und die Vorgehensweise der Arbeit dargelegt. Es soll kein Benutzerhandbuch zur eingesetzten Software enthalten! Die folgenden exemplarischen Gliederungen können als Orientierungshilfe dienen:

Beispiel für numerische Arbeit:
\begin{itemize}
\itemsep0em 
	\item Geometrie
	\item Netz
	\item Anfangs- und Randbedingungen
	\item Verwendete Solver, Zeitschrittweite, Konvergenzkriterien, ...
	\item Einlaufrechnung, Mitteilungsdauer, ...
\end{itemize}

Beispiel für experimentelle Arbeit:
\begin{itemize}
\itemsep0em 
	\item Versuchsaufbau und Geometrie
	\item Eingesetzte Messtechnik
	\item Angabe der Messgenauigkeit
	\item Vorgehensweise bei der Versuchsdurchführung
\end{itemize}

\section{Ergebnisse und Diskussion}

In diesem Kapitel werden die Ergebnisse beschrieben, diskutiert und ggf. mit anderen Daten aus der Literatur verglichen. Die Diskussion soll klären, ob die Daten sinnvoll sind, die Ergebnisse so zu erwarten waren und ob ungewöhnliche Beobachtungen sichtbar sind. Es ist auf eine übersichtliche Struktur zu achten, untersuchte Fälle sind konsistent und logisch zu benennen. Die Darstellung der Daten ist an die verwendete Messgenauigkeit anzupassen (z.B. ist eine Angabe von \SI{314.394575}{K} in den meisten Fällen nicht sinnvoll).

\section{Zusammenfassung und Ausblick}

Die Zusammenfassung ermöglicht Außenstehenden einen Überblick über den Inhalt der Arbeit und enthält insbesondere die Zielsetzung, die verwendete Methodik, Verfahren und Ansätze sowie die erzielten Ergebnisse. Daran schließt sich ein Ausblick an, der mögliche Verbesserungen bzw. weitere Arbeitsschritte aufzählt. Die Zusammenfassung ist das am häufigsten gelesene Kapitel (größt mögliche Sorgfalt!) und umfasst maximal zwei bis drei Seiten.

\section{Anhang}

Stören umfangreiche Ergebnisse oder Daten den Lesefluss, werden sie nur auszugsweise in den Kern der Arbeit übernommen. Der Rest kommt in den Anhang. Programmcode kommt nach Absprache mit dem Betreuer auch in den Anhang. Aktive Verweise auf den Anhang sind zu vermeiden. Nicht: \glqq In Abbildung A.12 ist eine Skizze des Versuchsaufbaus dargestellt. Die Probe ist unterhalb des Messkopfes eingespannt und ... \grqq. In diesem Fall ist die Abbildung zu wichtig für den Anhang und wird stattdessen besser im entsprechenden Kapitel platziert.
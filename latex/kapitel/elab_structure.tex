
\chapter{Project Structure and Requirements}

This chapter describes the structure of the project. First the overall organization and then the individual components are explained. Following that, the prerequisites and processes for building and/or running the code used in various parts of the project are briefly described.

\section{Requirements}
\label{SECTION:REQS}
This project requires a Linux system with {\itshape gcc}, {\itshape bash}, and a recent version of Python 3 with the {\itshape numpy} package to run the benchmarks. Additional Python packages are required to view and execute the analysis notebooks. {\itshape Jupyter Notebook}, {\itshape matplotlib},  {\itshape pandas}, and {\itshape seaborne} can be installed using the python package installer {\itshape pip}. Note: Jupyter requires at least Python 3.3.

\section{Directory Structure}
Figure \ref{FIG:folders} shows the file structure of the project.  First of all, the documentation (i.e.  this pdf) is in the {\itshape docs} folder, and the latex source files are in {\itshape latex}. The {\itshape ann\_1.2.2} directory contains the neighbor search C++ libraries that is examined. The {\itshape nanoflann} directory is an additional neighbor search library.  The folders {\itshape src} and {\itshape bin} contain the C++ source files (described in Section \ref{SECTION:SRC}) and their compiled binaries, respectively.  The {\itshape scripts} directory is home to the test case generation scripts (described in Section \ref{SECTION:TESTCASESCRIPTS}) and test execution scripts (see Chapter \ref{CHAPTER:BENCHMARKING}).  The {\itshape test} directory contains the working directories of the test runs and the result data.   Finally, the folder {\itshape analysis} contains the iPython notebooks that were used in the examination and analysis of the results.  These are described in more detail in Section \ref{SECTION:NOTEBOOKS}.

\begin{figure}[h]
	\centering
	\includegraphics[width=0.75\textwidth]{figures/project_folders.pdf}
	\caption{The root-level structure of the project repository.}
      \label{FIG:folders}
\end{figure}

\section{Search Methods}
\label{SECTION:SRC}

The nearest neighbor search methods are implemented in C++ 11. The {\itshape src} directory contains the source files. The linked-cell method is implemented in {\itshape fr\_cellLinkedList.cpp}. For the ANN method, there are multiple variants. First, {\itshape fr\_ann\_query.cpp} will only query a query single point for its neighbors. For the full nearest neighbor search, a method that builds the interaction pair lists ({\itshape fr\_ann.cpp}) and a method that skips this step ({\itshape fr\_ann\_nolist.cpp}) are provided. This was necessary due to the extremely long time required for generating the interaction pair lists.

The {\itshape src} directory also contains a Makefile which can be used to build any or all of the source files. The resulting executable files are placed in the {\itshape bin} directory upon a successful build. Finally, source files for the Nanoflann library are provided ({\itshape fr\_nanoflann.cpp} and {\itshape utils.h}) for possible future work evaluating this additional  nearest neighbor search method.

\section{Test Case Scripts}
\label{SECTION:TESTCASESCRIPTS}

To be able to evaluate the performance of the different nearest neighbor search algorithms, a number of different test cases are required. These take the form of a list of data points which are processed by the search method executable. The test cases differ by the distribution of the data points in the search domain and are described in detail in Section \ref{SECTION:TESTCASES}.

To generate the test cases, different scripts are used. Python was chosen as the scripting language due to familiarity and the availability of easy-to-use plotting libraries for visualization of the data point distributions. In the {\itshape scripts} directory, test case scripts are provided for four different distribution types examined in this work. With the exception of {\itshape test3d\_full}, which simply generates a filled domain, each of the scripts take certain parameters which control the fill and spacing of the points.  Note that due to the method used to generate the distributions, the fill factor passed to the script does not specify the fill directly - some experimentation is necessary to get the desired fill. 

The test case scripts are meant to be run within the benchmarking framework, as they also generate a statistics file for later use in the analysis. However, the scripts can also be run separately. Copy the script and the file {\itshape config.py} anywhere and execute the test case script with the Python interpreter. The data points are written to {\itshape data.pts} and can be visualized in 2D or 3D with scripts {\itshape plot\_2d.py} and {\itshape plot\_3d.py}.

\section{Jupyter Notebooks for Analysis}
\label{SECTION:NOTEBOOKS}

To examine and compare the test results, Python is again used in the form of Jupyter Notebooks found in the {\itshape analysis} directory. To access the notebooks, make sure to have the appropriate packages installed (see Section \ref{SECTION:REQS}). The Jupyter Notebook server can be started from the project root with the command $jupyter$ $notebook$ and notebooks are viewed and executed via a web browser.

The notebook files themselves are generally self explanatory. The overall structure of the analysis is as follows. In the data-preparation notebook, data is read from the results files in the {\itshape tests} directory, cleaned and labeled, and then written to a CSV file. The following notebooks read the nicely-formatted data from the CSV file and generate various tables and plots. In other words, if the data in the {\itshape tests} directory changes, the data-preparation notebook must be rerun in order to update the CSV file.

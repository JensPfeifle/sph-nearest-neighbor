%%%%%%%%%%%%%%%%%%%%%%%%%%%%%%%%%%%%%%%%%%%%%%%%%%%%%%%%%%%%%%%%%%%%%%%%%%%%%%%%%%%%%%%%
% Standardpakete
% Hilfe zu Paketen bieten die Kommandozeilentools texdoc (TeXLive) bzw. mthelp (Miktex)
%%%%%%%%%%%%%%%%%%%%%%%%%%%%%%%%%%%%%%%%%%%%%%%%%%%%%%%%%%%%%%%%%%%%%%%%%%%%%%%%%%%%%%%%
\usepackage[comma,
	sort,
% square,
%	numbers
	]{natbib}									% Formatierung des Literaturverzeichnisses 
\usepackage[ngerman,english]{babel} 			% Sprachpaket: neudeutsch, englisch, französisch
\usepackage[utf8]{inputenc}        				% Standardpackage: unterstützt Zeichentabellen
\usepackage[T1]{fontenc}              			% Standardpackage: unterstützt Schriftauswahl
\usepackage{lmodern}							% Schrift latin modern
\usepackage{amsmath}                  			% Standardpackage: Formelformatierung
\usepackage{amsfonts}                 			% Standardpackage: Formelformatierung
\usepackage{amssymb}                  			% Standardpackage: Formelformatierung
\usepackage{mathptmx}   						% Formeln und Text mit Adobe Times Roman Postscript-Schriften
\usepackage[scaled=.92]{helvet}					% Helvetica ist serifenlose Standardschriftart
\usepackage{courier}							% Courier ist typeset Standardschriftart
\usepackage{fixmath}   							% ISO-Formatierung griechischer Buchstaben
\usepackage[headsepline
	]{scrlayer-scrpage}							% Header-Formatierung
\usepackage{siunitx}                  			% Formatierung der Einheiten
\sisetup{
	per-mode=symbol,							% Form des Bruchstrichs: reciprocal (Exponent), symbol (/) oder fraction
	locale = DE 								% Ländereinstellung (Dezimaltrennzeichen usw.)
}
\usepackage{graphicx}							% Zusätzliche Optionen zur Grafikeinbindung
%\graphicspath{{./bilder/}]}

%%%%%%%%%%%%%%%%%%%%%%%%%%%%%%%%%%%%%%%%%%%%%%%%%%%%%%%%%%%%%%%%%%%%%%%%%%%%%%%%%%
% Zusätzliche Pakete
%%%%%%%%%%%%%%%%%%%%%%%%%%%%%%%%%%%%%%%%%%%%%%%%%%%%%%%%%%%%%%%%%%%%%%%%%%%%%%%%%%
\usepackage[
	margin=10pt,
	labelformat=simple,
	labelsep=colon,
	font=normal,
	labelfont=normal,
	skip=15pt
	]{caption}[2008/08/24]  					% Erlaubt die Formatierung der Bildunter- und Tabellenüberschriften
\usepackage{ifthen}								% Paket für Abfragen
\usepackage{color}								% Farbige Texte
\usepackage{transparent}						% Transparenz
\usepackage{booktabs}							% Nützliche Tabellentools (\topline, \midline, \bottomline, ...)
\usepackage{longtable}							% Tabellen über mehrere Seiten

%%%%%%%%%%%%%%%%%%%%%%%%%%%%%%%%%%%%%%%%%%%%%%%%%%%%%%%%%%%%%%%%%%%%%%%%%%%%%%%%%%
% Optionale (aber evtl. nützliche) Pakete
%%%%%%%%%%%%%%%%%%%%%%%%%%%%%%%%%%%%%%%%%%%%%%%%%%%%%%%%%%%%%%%%%%%%%%%%%%%%%%%%%%
\usepackage{lipsum}								% Einbindung von Lorem Ipsum als Platzhalter
\usepackage{multirow}                 			% Befehle analog \multicolumn in vertikaler Richtung
\usepackage{subcaption}							% Ermöglicht die Verwendung von subfigure
\PreventPackageFromLoading{subfig} 				% Verhindert das Laden des veralteten Packages subfig
\usepackage[svgpath=bilder/]{svg}	  			% Automatisches Einbinden von svg-Dateien
\usepackage{textcomp}							% Ermöglicht die Einbindung von Symbolen
\usepackage[
	colorlinks=true, 
	linkcolor=black, 
	citecolor=black, 
	urlcolor=black
	]{hyperref}									% Verlinkte Referenzen
\hypersetup{
	pdftitle={\doctitle},
	pdfsubject={\docsubject},
	pdfauthor={\docauthor},
	pdfkeywords={\dockeywords},
  pdfcreator={\doccreator},
  pdfproducer={\docproducer}
}
\usepackage{nameref}							% Definition der Metadaten, die dem PDF-Dokument hinterlegt werden
\usepackage[nonumberlist,
	nomain=true,
	acronym,
	toc,
	numberedsection=false,
	]{glossaries}								% Glossaries-Paket zur Erstellung eines Glossars
\newcounter{dummy} 								% korrekte Verlinkung der Literatur und anderer Verzeichnisse